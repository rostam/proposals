\documentclass[conference]{IEEEtran}
\title{Improving Cross Cultural Communications Using Serious Games
Based on Cultural-Oriented Adaptative Scenarios}
\date{June 8, 2014}
\author{Masoumeh Seydi\\ Department of Computer Science\\
       Konstanz University, Germany}

\usepackage{graphicx}
\begin{document}
\maketitle

\section{Introduction}

In the wide area of social communications, the cross-cultural communication is 
a major topic. Human beings might behave differently in communications with
new cultures. The communications can be more serious when the group or society 
follow a goal and the cultural diversity might lead to damaging behaviors and communications,
the group plans or even the goal would be negatively affected.
There are a lot of research in this area.
For example, the international company with employees from various nationalities
and cultural backgrounds, need a healthy and strong communications to benefit the most potential
of the group.
This could apply to the other societies in any size, such as family, group of student that
are doing an academic project, professional working groups, dormitories, and etc. 

Through the state of the art concepts in serious gaming~\cite{sergame1,sergame2}, 
there are noticeable efforts to use it for the communication purposes as well. (Ref needed)
\footnote{We mean the general definition of the serious games as 
usage of games to simulate the serious situations.}
Like any other type of communications in the literature, serious games could be applied 
to the cross cultural communication.
The TARGET project~\cite{project-manage} is a good example by which cross cultural issues in 
project managements is studied and analysied into a serious game framework.
Although TARGET is a powerful platform toward cross cultural communication,
it is not easily adaptive for general purpose areas.

To analyse these cross cultural effects, we need
a series of complicated real situations, namely scenarios.
Although there is a long tradition in generating a game
for a specific purpose and specific scenario, 
there is still room for research 
to find a more general purpose framework for various scenarios.

In this proposal, we aim to design a serious game platform which helps to 
improve the cross cultural communications based on the social behavior studies.
This platform would be preferably flexible to adapt to new different situations coming
from various ways of scenarios. 

\section{Research on Cross Cultural Communication}

Studying and classifying the 
communications between the people with few or none understanding or familiarity are the main
starting points. In order to implement 
more real experiences, the platform should be flexible enough to define the 
various social scenarios in the extent of our study.
Suppose social events and communications are formed by some basic components,
\begin{itemize}
\item One or more individual
\item One or more goal
\item A scenarios including a set of happenings and activities to achieve the goal(s).
\end{itemize}
As an example, in a shared apartment in which the flat-mates are from culturally different background, each flatmate would have its own lifestyle, social contacts, cookings, and activities. The differences might suppress the activities or either disturb the other people physically or mentally. If people have a chance to practice the new environment in a virtual space, the can be familiar with situations in which they either face new cultural concepts or they represent their individual cultural concepts. This could help to estimate the individual tolerance in both situations and to form familiarity with new cultural view. 
The platform should allow to define these components of various events and communications in 
arbitrary scales to form group connections and activities. It means that the character are able to define the scenarios which follow a specific goal
The platform can be used to define various cross-cultural environments in order to
provide virtually simulated real experiences. The characters are virtually exposed to culturally
different contacts, vocal and facial expressions, events, body language, and etc.
The communications should be verbal to give the more feeling of reality. 

%. [projectmanage] describes a sample of society which needs healthy and strong communications.

\section{Scenario}
\cite{scenario}
Here, we discuss more about the scenario structure which can be
an interesting research topic during the work.
Although not all stories follow a similar structure,
the general classifications should be studied. The goal of such a study is to 
formulate a general model for stories and senarios.
We can think first to have a restricted structure for a scenario.
Then, we can do one of the following approaches: 
\begin{itemize}
\item Generate a new scenario based on this structure, like~\cite{scenario-gen} or \cite{l-system}.
\item Adapt an existing story to our structure, like~\cite{scenario-adapt}
\item Repurposing an existing scenario, like~\cite{scneario-repurposing}
\end{itemize}

We can restrict our definition of the scenario as a story which
can be completely described with the following basic components:
\begin{itemize}
\item Characters: one or more actual persons as well as many virtual persons
\item Breakpoints: cuts in scenario which can be assumed as small goals 
\item Final goals: it is somehow the end of the game and 
it can be various from a robbery to a treasure
\item Puzzles: a riddle or any other kinds of puzzle which should be solved in each 
      breakpoint and in the big goal.
\item Distance between Breakpoints: a specific duration of the time.
\end{itemize}

\section{Platform}
Starting from the scenarios designed based on the previous discussion,
the target group can have a serious game to assess the social behaviour.
The processes of the game is to reach the intermediate goals in the given time.
Communications toward the goal is byby providing online of offline sound, image, and text messages either. 
It means that for each point in a process the players can either contact each other directly or save a message. 
Then, the others could read or hear it when they want. The message could be specified to a location. In this case, 
the messages will be shown as a hint in that location when the players arrive to that location. These messages can be hints or information about the path or the goal they are following.

We combine the following research to apply the scenario:
\begin{itemize}
\item Learning based on the location-based games like the ideas in ~\cite{rexplorer} and 
a large scale~\cite{scal-loc-g}.
\item The ability of the sensors in mobile phones like the ones explained in~\cite{sensors}.
\item Adapting the scenarios and generating the breakpoints
\item Attachment of game elements to each breakpoint (for a try in this direction 
look at \cite{totem})
\end{itemize}


Each breakpoint include one or more of the following elements:
\begin{itemize}
\item Location: GPS location
\item Sound: Recorded sound from virtual characters or a sound
recorded by a teammate
\item Image and Video
\item Contacts
\item Messages
\item Augmented Reality: specific objects floating in the space which is 
visible only from the mobile phone (it is not actually there in the real world).
It can be like a corpse in the scene of a specific scenario.
\end{itemize}


A research is also needed regarding the human interaction and communication
together and with the device, like \cite{behavior} and \cite{facial-vocal}.


%-----------------------------------------------------------------
\begin{figure}
 \centering
\includegraphics[width=0.14\textwidth]{kar}
\caption{The Starting page of mobile (Prototype}
\label{diagram}
\end{figure}
%-----------------------------------------------------------------


\bibliographystyle{unsrt}
\bibliography{refs}
\end{document}
