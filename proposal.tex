\documentclass[conference]{IEEEtran}
\title{Improving Cross Cultural Communications Using Serious Game
Based on Cultural-Oriented Scenarios}
\date{June 8, 2014}
\author{Masoumeh Seydi\\ Department of Computer Science\\
       Konstanz University, Germany}

\usepackage{graphicx}
\begin{document}
\maketitle

\section{Introduction}

In the wide area of social communications, the cross-cultural communication is 
a significant topic. Human beings might behave differently when facing
new cultures. Besides, lack of understanding cultural differences and their impacts can cause conflicts. The difficulties can be more serious when a group or society 
follow a goal and the cultural diversity might negatively affect 
the group plans and the goal. 
The globalisation phenomena has brought cosmopolitan mix of different cultures. Thereby, individuals need to overcome culture shocks and bridge the cross-cultural differences.
%There are a lot of research in this area.
For example, an international company with employees from various nationalities
and cultural backgrounds, need a healthy and strong communications to benefit the most potential
of the group.
This applies also to other societies in any size, such as families, group of students in an academic project, professional working groups, dormitories, and etc. 

Through the state of the art concepts in serious game~\cite{sergame1,sergame2}, 
there are noticeable efforts to use it for the communication purposes as well~\cite{comm-sergame}.
\footnote{We mean the general definition of the serious games as 
usage of games to simulate the serious situations.}
Like any other type of communication in the literature, serious games could be also applied 
to the cross cultural communication.
The TARGET project~\cite{project-manage} is a good example by which cross cultural issues in 
project management is studied and analyzed into a serious game framework.
Although TARGET is a powerful platform in this topic,
it is not easily adaptive for general-purpose areas.

To analyze the cross cultural effects, we need
a series of complicated real situations, namely scenarios.
Although there is a long tradition in generating a game
for a specific purpose and scenario, 
there is still room for research 
to find a more general framework for various cross cultural scenarios.

In this proposal, we aim to design a serious game platform which helps to 
improve the cross cultural communications based on the social behavior studies.
This platform should be flexible enough to provide 
defining various scenarios in the extent of our study. 
The platform can be used to define various cross-cultural environments in order to
provide virtually simulated real experiences. The characters are virtually exposed to culturally
different contacts, vocal and facial expressions, body language, and etc.
The communications should be verbal to give the more feeling of reality. 
This platform is used to generate serious games for mobile phones.

\section{Research on Cross Cultural Communication}
Studying the communications between the people,
which may not be familiar with each other's cultures, is the main
starting point. 
Our research will help to find and classify first the general cultural types 
and then the problems which might cause 
communicational problems and conflicts. 
These problems can be formulated as different scenarios.
As an example, in a shared apartment in which the flat-mates are from culturally different background, each flatmate would have its own lifestyle, social contacts, and activities. The differences might suppress the activities, disturb the other people physically or mentally or lead to a cultural shock. Another example is the communications of the children and teenagers in schools which can be highly affected by cultural diversity. 
%In order to implement 
%more real experiences, the platform should be flexible enough to define the 
%various social scenarios in the extent of our study.

Starting to formulate these conflicts, we should discuss scenarios.
Suppose social scenarios are formed by some basic components,
\begin{itemize}
\item One or more individual
\item One or more goal
\item A scenarios including a set of activities to achieve the goal(s) 
in a specific time.
\end{itemize}
Based on the social scenarios, different groups and their connections are formed in arbitrary scales. 
Regarding a suitable behavior analysis, we prefer those scenarios which address more cultural differences and fit to the chosen group. Later, the scenarios can be used in forming a serious game by which the cross-cultural challenges can be played to some extent. Since the virtual setting let the people to be themselves, the difficulties are truly shown when they specify its own activities and react to other activities or happenings.

For example, in a group of students working on a semester project, the type of communications and behaviors to plan a working schedule can be counted as scenarios. One might prefer offline and written communications while the other might prefer verbal communication and face-to-face discussions. The goal of a serious game here is to finish the semester project. The scores also can be related to the quality of the project. We can use scores from the game to analyze the behavior of this group. 

%. [projectmanage] describes a sample of society which needs healthy and strong communications.

\section{Platform}
Based on the social studies in this project, we design services and components to implement real-world settlements using mobile phones. A network of individuals will have a chance to engage with the new environment and face cross-cultural challenges. This virtual space makes them familiar with situations in which they either face new cultural concepts or they represent their individual ones. This could help to achieve a higher level of orientation and to estimate the individual tolerance in both situations. In this way, social behavior assessment is achievable and the core competency of a persons which lacks can be identified. Then, the chance of communication developments increases by knowing more about the problems.

Our platform provides a user interface in which anyone can submit their scenario.
This helps to improve the originality of the scenario regarding the cultural issues.
To manage this user interface we need to have a specific structure for the scenario
which we discuss in the next section.

Starting from the scenarios designed based on the previous discussion,
the target group can have a serious game to assess the social behavior.
The processes of the game is to reach the intermediate goals in the given time.
Communications toward the goal is done by providing mainly online verbal, facial, and physical expressions and offline sound, image, and text messages in specific case. 

Toward this goal, we need to tackle different challenges. In the next section, we discus
some of these challenges and how to overcome. Also, we do a brainstorming to gather ideas
how such a platform would look like.

\section{Research Challenges}
Perhaps, the first problem would be how to manage scenarios
which can be submitted to our platform.
Although not all scenarios follow a similar structure,
a general classification should be studied. 
The goal of such a study is to 
formulate a general model for scenarios.
We can think first to have a restricted structure for a scenario
while the scenario remain exciting~\cite{scenario}.
Having this restricted structure, we can generate a new scenario~\cite{scenario-gen,l-system},
adapt an existing scenario~\cite{scenario-adapt}, 
or repurposing an existing scenario~\cite{scenario-repurposing}.

In a scenario, the processes of game would be 
to reach the intermediate goals in the given time.
We should combine various research directions for developing such a game.
The engagement of in the multi-cultural atmosphere is one of the first steps,
like~\cite{rexplorer, scal-loc-g}. 
Attachment of game elements, like the location, images, videos, and augmented reality, to each intermediate goal is the next step.
TOTEM~\cite{totem} is one of the interesting research in this area.
Finally, a research is also needed regarding the human interaction and communication
together and with the device, like \cite{behavior} and \cite{facial-vocal}.
This kind of research is needed an intensive work together with the people 
from the social studies.

\bibliographystyle{unsrt}
\bibliography{refs}
\end{document}
